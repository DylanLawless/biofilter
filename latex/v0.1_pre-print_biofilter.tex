% to compile: \ll

%%%%%%%%%%%%%%%%%%%%%%%%%%%%
%%%% Document type      %%%%
%%%%%%%%%%%%%%%%%%%%%%%%%%%%

\documentclass[preprint,11pt,fleqn]{elsarticle}
\usepackage{nopageno} % no page numbers

    %%%%%%%%%%%%%%%%%%%%%%%%%%%%
    %%%% Document structure %%%%
    %%%%%%%%%%%%%%%%%%%%%%%%%%%%
    \usepackage[margin=3.6cm]{geometry} % boarder size
    \usepackage{lineno} % used along with \linenumbers after begin document. 
    \usepackage{setspace} % line spacing
        \setstretch{1.4}
    \makeatletter % The following lines get rid of footer stating pre-preint to elsevier.
    	\def\ps@pprintTitle{%
    	\let\@oddhead\@empty
    	\let\@evenhead\@empty
    	\def\@oddfoot{}%
    	\let\@evenfoot\@oddfoot}
    \makeatother
    \graphicspath{ {images/} } % sets the path to image files (Figures)
    
    %%%%%%%%%%%%%%%%%%%%%%%%%%%%
    %%%% Bibliography       %%%%
    %%%%%%%%%%%%%%%%%%%%%%%%%%%%
    %\usepackage{natbib} \setcitestyle{comma,authoryear}
    \usepackage{natbib}
    	\setcitestyle{numbers,sort&compress}
    	\setcitestyle{sort&compress}
    	\usepackage{hypernat} % hypernat also required to allow citations to compress. 
    
    %%%%%%%%%%%%%%%%%%%%%%%%%%%%
    %%%% Aesthetics         %%%%
    %%%%%%%%%%%%%%%%%%%%%%%%%%%%
    \usepackage{microtype} % Creates better spaced text
    \RequirePackage{times} % Font
    \usepackage{ccaption}
    \usepackage{siunitx} % SI units 
    \usepackage[T1]{fontenc}
    \usepackage[utf8]{inputenc}
    \usepackage{nameref}% this allows a reference be named, to print unnumbered references by their section name (used here for linking to Supplemental text in this case).
    \usepackage{xcolor} % Setting colours and their usage
    	\definecolor{natureblue}{RGB}{5,110,210}
        \usepackage[colorlinks]{hyperref} % Colour for hyperlinks, (URLs, citations, cross reference)
    	\AtBeginDocument{%this allows colours to chage from the defined elsearticle template.
    	\hypersetup{
    	colorlinks=true,
        linkcolor={natureblue},
    	citecolor={natureblue},
        filecolor=blue!50!black,
        urlcolor=cyan,
    	}}

    %%%%%%%%%%%%%%%%%%%%%%%%%%%%
    %%%% Supplemental labels%%%%
    %%%%%%%%%%%%%%%%%%%%%%%%%%%%
    %Define command to start a supplemental section
    %set the supplemental letter used for figures (e.g. Figure E1)
    \newcommand{\beginsupplement}{%
            \setcounter{table}{0}
            \renewcommand{\thetable}{E\arabic{table}}%
            \setcounter{figure}{0}
            \renewcommand{\thefigure}{E\arabic{figure}}%
         }
    
    %%%%%%%%%%%%%%%%%%%%%%%%%%%%
    %%%% Building tables    %%%%
    %%%%%%%%%%%%%%%%%%%%%%%%%%%%
    \usepackage{booktabs} % required for tables
    \usepackage{rotating,tabularx} % tabularx is the table style used, rotating can also be used
    \newcolumntype{Z}{ >{\centering\arraybackslash}X } % defining table content layout per box
    \usepackage{ltablex} % allow page break between lines in tabularx
    \usepackage{caption} \captionsetup{font=normalsize} % to set the caption size as normal even when table is tiny.
    \usepackage{multirow}
    \usepackage{pdflscape}


%%%%%%%%%%%%%%%%%%%%%%%%%%%%
%%%% Begin content      %%%%
%%%%%%%%%%%%%%%%%%%%%%%%%%%%

\begin{document}
%  \linenumbers
\small

%%%%%%%%%%%%%%%%%%%%%%%%%%%%
%%%% Title and author   %%%%
%%%%%%%%%%%%%%%%%%%%%%%%%%%%
\begin{frontmatter}

\title{Genomic and biochemical potential in transporter biofilter technology}
    \author[add1]{Placeholder title \corref{cor1}}
\author[add1]{Dylan Lawless\corref{cor1}}
	\ead{Dylan(dot)Lawless(at)epfl(dot)ch}
\cortext[cor1]{Addresses for correspondence}
\address[add1]{Global Health Institute, School of Life Sciences, École Polytechnique Fédérale de Lausanne, Lausanne, 1015, Switzerland}

%%%%%%%%%%%%%%%%%%%%%%%%%%%%
%%%% Abstract           %%%%
%%%%%%%%%%%%%%%%%%%%%%%%%%%%

\begin{abstract}
\setstretch{1.2}
Abstract
\end{abstract}
\end{frontmatter}

%%%%%%%%%%%%%%%%%%%%%%%%%%%%
%%%% Formalities        %%%%
%%%%%%%%%%%%%%%%%%%%%%%%%%%%
 %   \section*{Funding}
 %   \section*{Acknowledgements}
 %   \section*{Ethics statement}
 %   \section*{Key words}
    \section*{Abbreviations}
    \noindent 
   % ABC (ABC)

%%%%%%%%%%%%%%%%%%%%%%%%%%%%
%%%% Introduction       %%%%
%%%%%%%%%%%%%%%%%%%%%%%%%%%%
% \tableofcontents
% \listoffigures
%  \listoftables

\section{Introduction}
\noindent

\section{Screening complexity}
Whitelists, blacklists, and subject-specific definition. 


\section{Genomic screen}
Tolerated genomic sequences.
A benign microbe may acquire pathogenic sequences, plasmids.

\section{Peptide screen}

What is the number of unique proteins and peptide encoded by host genome.
What is the number of commensal organisms.


\section{Immonogenic tolerance filter}
The distance between foreign peptide and host petides may indicate MHC presentation.
MHC presented unknown peptides may ilicite a flag - do not want to impede the normal host immune response. 
If flagged peptide producer can be identified it can be filtered as pathogen.
If not then send for further screen.

If peptides are too far from host cell or component then foreign material may be flagged.

How do we define commensal organisms. 
A subject specific immune profile should be recorded - HLA and somatic recombination matrix. 
We can predict the expected binding potential based on germline and somatic recombination of the HLA. 

\section{Decision tracing patter}
A fixed decission tree allows for learned patterns.


\section{Parasitic commensals}
Some organisms may be more paracitic than commensal, however, these apparent
pathogens may infact provide beneficial compouds, hormones, or keep more damaging pathogens in check - commensal antibiotic. 
What criteria would save these organisms when all others would flag them for filtering?

Obligate parasites may need to be transferred to a synthetic host before assessing for extermination. 
The rules for sentience must be applied. 



%%%%%%%%%%%%%%%%%%%%%%%%%%%%
%%%% Conclusion         %%%%
%%%%%%%%%%%%%%%%%%%%%%%%%%%%

\section{Conclusion}

\section*{Authorship Contributions}
%\noindent Dylan Lawless analysed data and conceived and wrote the manuscript, 
Filler content.

%%%%%%%%%%%%%%%%%%%%%%%%%%%%
%%%% Conflict of Interest%%%%
%%%%%%%%%%%%%%%%%%%%%%%%%%%%

\section*{Conflict of Interest}
\noindent The authors declare no conflict of interest.

\clearpage

%%%%%%%%%%%%%%%%%%%%%%%%%%%%
%%%% References         %%%%
%%%%%%%%%%%%%%%%%%%%%%%%%%%%

\section*{\refname}
	\bibliographystyle{unsrtnat}
%	\bibliographystyle{plainnat}
	\bibliography{references}
\pagebreak 


%%%%%%%%%%%%%%%%%%%%%%%%%%%%
%%%% Supplemental       %%%%
%%%%%%%%%%%%%%%%%%%%%%%%%%%%
\beginsupplement
\section{Supplemental}
\label{sec:Supplemental_text}

% \begin{figure}[h]
%	\hspace*{0cm} \includegraphics[scale=0.62]{Supp_fig}
%	\caption{Caption filler}
%	\label{fig:Supp_fig}
%\end{figure}
%\clearpage

\end{document}
%%%%%%%%%%%%%%%%%%%%%%%%%%%%
%%%% End of Document    %%%%
%%%%%%%%%%%%%%%%%%%%%%%%%%%%
