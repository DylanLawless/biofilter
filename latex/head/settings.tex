
    %%%%%%%%%%%%%%%%%%%%%%%%%%%%
    %%%% Document structure %%%%
    %%%%%%%%%%%%%%%%%%%%%%%%%%%%
    \usepackage[margin=3.6cm]{geometry} % boarder size
    \usepackage{lineno} % used along with \linenumbers after begin document. 
    \usepackage{setspace} % line spacing
        \setstretch{1.4}
    \makeatletter % The following lines get rid of footer stating pre-preint to elsevier.
    	\def\ps@pprintTitle{%
    	\let\@oddhead\@empty
    	\let\@evenhead\@empty
    	\def\@oddfoot{}%
    	\let\@evenfoot\@oddfoot}
    \makeatother
    \graphicspath{ {images/} } % sets the path to image files (Figures)
    
    %%%%%%%%%%%%%%%%%%%%%%%%%%%%
    %%%% Bibliography       %%%%
    %%%%%%%%%%%%%%%%%%%%%%%%%%%%
    %\usepackage{natbib} \setcitestyle{comma,authoryear}
    \usepackage{natbib}
    	\setcitestyle{numbers,sort&compress}
    	\setcitestyle{sort&compress}
    	\usepackage{hypernat} % hypernat also required to allow citations to compress. 
    
    %%%%%%%%%%%%%%%%%%%%%%%%%%%%
    %%%% Aesthetics         %%%%
    %%%%%%%%%%%%%%%%%%%%%%%%%%%%
    \usepackage{microtype} % Creates better spaced text
    \RequirePackage{times} % Font
    \usepackage{ccaption}
    \usepackage{siunitx} % SI units 
    \usepackage[T1]{fontenc}
    \usepackage[utf8]{inputenc}
    \usepackage{nameref}% this allows a reference be named, to print unnumbered references by their section name (used here for linking to Supplemental text in this case).
    \usepackage{xcolor} % Setting colours and their usage
    	\definecolor{natureblue}{RGB}{5,110,210}
        \usepackage[colorlinks]{hyperref} % Colour for hyperlinks, (URLs, citations, cross reference)
    	\AtBeginDocument{%this allows colours to chage from the defined elsearticle template.
    	\hypersetup{
    	colorlinks=true,
        linkcolor={natureblue},
    	citecolor={natureblue},
        filecolor=blue!50!black,
        urlcolor=cyan,
    	}}

    %%%%%%%%%%%%%%%%%%%%%%%%%%%%
    %%%% Supplemental labels%%%%
    %%%%%%%%%%%%%%%%%%%%%%%%%%%%
    %Define command to start a supplemental section
    %set the supplemental letter used for figures (e.g. Figure E1)
    \newcommand{\beginsupplement}{%
            \setcounter{table}{0}
            \renewcommand{\thetable}{E\arabic{table}}%
            \setcounter{figure}{0}
            \renewcommand{\thefigure}{E\arabic{figure}}%
         }
    
    %%%%%%%%%%%%%%%%%%%%%%%%%%%%
    %%%% Building tables    %%%%
    %%%%%%%%%%%%%%%%%%%%%%%%%%%%
    \usepackage{booktabs} % required for tables
    \usepackage{rotating,tabularx} % tabularx is the table style used, rotating can also be used
    \newcolumntype{Z}{ >{\centering\arraybackslash}X } % defining table content layout per box
    \usepackage{ltablex} % allow page break between lines in tabularx
    \usepackage{caption} \captionsetup{font=normalsize} % to set the caption size as normal even when table is tiny.
    \usepackage{multirow}
    \usepackage{pdflscape}
